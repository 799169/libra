\documentclass{jsarticle}
\usepackage{amsmath,amssymb,amsthm,ascmac,enumerate,framed}

\everymath{\displaystyle}
\DeclareMathOperator{\Mod}{mod}
\newcommand{\N}{\mathbb{N}}
\newcommand{\Z}{\mathbb{Z}}

\theoremstyle{definition}
\newtheorem*{Dfn}{定義}
\newtheorem*{Exm}{例}
\newtheorem{Prp}{命題}
\renewcommand{\proofname}{\textbf{証明}}
\newenvironment{dfn}{\vspace{1ex}\begin{screen}\begin{Dfn}}{\end{Dfn}\end{screen}\vspace{1ex}}
\newenvironment{exm}{\begin{leftbar}\begin{Exm}}{\end{Exm}\end{leftbar}}
\newenvironment{prp}{\vspace{1ex}\begin{screen}\begin{Prp}}{\end{Prp}\end{screen}}
\newenvironment{prf}{\begin{leftbar}\begin{proof}}{\end{proof}\end{leftbar}}

\title{メモ (形式的冪級数)}
\author{hos}

\begin{document}
\maketitle

大事なこと:
収束のことは考えない.
その代わり,知らない演算をしない.

$\N$ は非負整数全体の集合とする\footnote{普段は $\Z_{\ge 0}$ とかを使って $\N$ という記号を避けようと思っているのですが,$\sum$ の下にたくさん書くので仕方なく.}.

\section{形式的冪級数環}
$R$ を可換環とする\footnote{環と言ったら乗法の単位元の存在を仮定します.}.

$X$ を不定元として,
$R$ の加算個の直積 $\prod_{i\in\N} R$ の元 $(a_i)_{i\in\N}$ を形式的に
$\sum_{i\in\N} a_i X^i$ (あるいは $a_0 + a_1 X + a_2 X^2 + \cdots$) と書いたものの集合を $R[[X]]$ とする.
この元を $a(X)$ のように書くこともある.
$a_i$ を $a(X)$ の $i$ 次の\textbf{係数} (あるいは $X^i$ の係数) と呼び,$[X^i] a(X)$ のように書く.
$0$ 次の係数を\textbf{定数項}と呼ぶ.

$\sum_{i\in\N} a_i X^i, \sum_{i\in\N} b_iX^i \in R[[X]]$ に対し,加法と乗法を
\begin{align*}
  \sum_{i\in\N} a_i X^i + \sum_{i\in\N} b_i X^i &:= \sum_{i\in\N} (a_i + b_i) X^i, \\
  \left(\sum_{i\in\N} a_i X^i\right) \left(\sum_{i\in\N} b_i X^i\right) &:= \sum_{i\in\N} \left(\sum_{i,j\in\N,\,i+j=k} a_i b_j\right) X^k
\end{align*}
で定めると,可換環になることを示す.
加法の単位元は $0 = 0 + 0 X + 0 X^2 + \cdots$,
乗法の単位元は $1 = 1 + 0 X + 0 X^2 + \cdots$.
乗法の結合法則のみやや非自明で,
$\left(\left(\sum_{i\in\N} a_i X^i\right) \left(\sum_{i\in\N} b_i X^i\right)\right) \left(\sum_{i\in\N} c_i X^i\right)$ の
$m$ 次の係数が
\[
  \sum_{l,k\in\N,\,l+k=m} \left(\sum_{i,j\in\N,\,i+j=k} a_i b_j\right) c_k = \sum_{i,j,k\in\N,\,i+j+k=m} a_i b_j c_k
\]
となることから従う.
この可換環 $R[[X]]$ を,\textbf{$R$ 係数形式的冪級数環}と呼ぶ.

さらに,$r \in R$ はそのまま $r + 0 X + 0 X^2 + \cdots \in R[[X]]$ とみれるので,
包含 $R \lhook\joinrel\longrightarrow R[[X]]$ により $R[[X]]$ は $R$ 代数でもある.

\begin{exm}
  $(1 + X) (1 + X + X^2 + X^3 + \cdots) = 1 + 2 X + 2 X^2 + 2 X^3 + \cdots$.
\end{exm}

$a(X) \in R[[X]]$ に対し,
集合 $a(X) R[[X]] := \{ a(X) b(X) \mid b(X) \in R[[x]] \} \subseteq R[[X]]$ は
$R[[X]]$ のイデアルである.
$b(X), c(X) \in R[[X]]$ が $b(X) - c(X) \in a(X) R[[X]]$ を満たすことを
$b(X) \equiv c(X) \pmod{a(X)}$ と書く.
$n \in \N$ に対し,$\Mod X^n$ での合同は,$n$ 次未満の係数が等しいことを表す.

\begin{exm}
  $0 + 1 X + 2 X^2 + 3 X^3 + 4 X^4 + \cdots \equiv X + 2 X^2 \pmod{X^3}$.
\end{exm}

次の命題は,突き詰めると環の位相の話になるが,本稿では技術的な補題として用いる.

\begin{prp}
  \label{prp:mod-limit}
  $a(X), b(X) \in R[[X]]$ について,
  $a(X) = b(X)$ である必要十分条件は,任意の $n \in \N$ に対して $a(X) \equiv b(X) \pmod{X^n}$ であること.
\end{prp}
\begin{prf}
  (必要性)
  明らか.

  (十分性)
  任意の $i \in \N$ に対し,$n = i + 1$ ととって $a(X) \equiv b(X) \pmod{X^{i+1}}$ なので,$a_i = b_i$ となる.
\end{prf}

\section{乗法の逆元}
環の単元とは,乗法の逆元をもつ元のこと.可逆元.$1$ の約数.

\begin{prp}
  $\sum_{i\in\N} a_i X^i \in R[[X]]$ が単元であるための必要十分条件は,$a_0$ が $R$ の単元であること.
\end{prp}
\begin{prf}
  (必要性)
  $\sum_{i\in\N} b_i X^i \in R[[X]]$ が $\left(\sum_{i\in\N} a_i X^i\right) \left(\sum_{i\in\N} b_i X^i\right) = 1$ を満たすとすると,
  定数項を比較して,$a_0 b_0 = 1$ である.

  (十分性)
  $a_0$ が単元のとき,$\sum_{i\in\N} b_i X^i \in R[[X]]$ を
  \begin{align*}
    b_0 &= a_0^{-1}, \\
    b_i &= -a_0^{-1} \sum_{j\in\N,\,1\le j\le i} a_j b_{i-j} \quad (i \ge 1)
  \end{align*}
  として定めると,$\left(\sum_{i\in\N} a_i X^i\right) \left(\sum_{i\in\N} b_i X^i\right) = 1$ を満たす.
\end{prf}

$a(X) \in R[[X]]$ の乗法の逆元が存在するとき,それは一意なので,
$a(X)^{-1}$ や $\frac{1}{a(X)}$ と書く.

\begin{exm}
  $r \in R$ に対し,$(1 - r X)^{-1} = \sum_{i\in\N} r^i X^i$.
\end{exm}

ここまでで定めた加減乗除については,一般の $R$ 代数で成り立つことを用いて普通の計算ができるし,普通の表記をする.

\begin{exm}
  $a(X) \in R[[X]]$ に対して,$a(X)^2$ とは $a(X) a(X)$ のことであり,
  $a(X)^2$ の逆元は $(a(X)^{-1})^2$ であり $a(X)^{-2}$ と書く.
\end{exm}

\begin{exm}
  正の整数 $n$ に対し,$(1 - X)^{-n} = \sum_{i\in\N} \binom{i+n-1}{n-1} X^i$.
\end{exm}


\section{合成}
形式的冪級数の合成は,$X$ の部分に「代入」していいものは定数項が $0$ でなければならないことに注意を要する.

\begin{dfn}
  $a(X) = \sum_{i\in\N,\,i\ge 1} a_i X^i \in X R[[X]]$ および
  $b(X) = \sum_{i\in\N} b_i X^i \in R[[X]]$ に対し,
  $b(X)$ と $a(X)$ の\textbf{合成} $(b \circ a)(X)$ を
  \[
    % (a \circ b)(X) := \sum_{i\in\N} \left(\sum_{j\in\N,\, k_1,\ldots,k_j\in\N,\, k_1,\ldots,k_j\ge 1,\, k_1+\cdots+k_j=i} a_j b_{k_1} \cdots b_{k_j} \right) X^i
    (b \circ a)(X) := \sum_{i\in\N} \left(\sum_{k\in\N,\, j_1,\ldots,j_k\in\N,\, j_1,\ldots,j_k\ge 1,\, j_1+\cdots+j_k=i} b_k a_{j_1} \cdots a_{j_k} \right) X^i
  \]
  で定める.
  内側の $\sum$ について,$k \le i$ が従うためこれは有限和である.
  特に,$(b \circ a)(X)$ の定数項は $a_0$ である.
\end{dfn}

$(b \circ a)(X)$ の $i$ 次の係数は,$b_k a(X)^k$ の $i$ 次の係数を $k \in \N$ について足したものである.
つまり,形式的に $(b \circ a)(X) = \sum_{k\in\N} b_k a(X)^k$ と書きたいが,
右辺は $R[[X]]$ の元の無限和であり定義されておらず,各係数ごとに有限和として定義できるための条件が $a(X)$ の定数項が $0$ であることに他ならない.
このとき,$k > i$ の項は $i$ 次の係数に影響を与えない.
言い換えると,
\begin{prp}
  \label{prp:composition-mod}
  $a(X) = \sum_{i\in\N,\,i\ge 1} a_i X^i \in X R[[X]]$ および
  $b(X) = \sum_{i\in\N} b_i X^i \in R[[X]]$ に対し,
  \[
    (b \circ a)(X) \equiv \sum_{k\in\N,\,k<n} b_k a(X)^k \pmod{X^n}
  \]
  が成り立つ.
\end{prp}
\begin{prf}
  $i \in \N$, $i < n$ に対し,$k \le i$ ならば $k < n$ であるから,
  \begin{align*}
    % (a \circ b)(X)
    % &\equiv \sum_{i\in\N,\,i<n} \left(\sum_{j\in\N,\, k_1,\ldots,k_j\in\N,\, k_1,\ldots,k_j\ge 1,\, k_1+\cdots+k_j=i} a_j b_{k_1} \cdots b_{k_j} \right) X^i \\
    % &= \sum_{i\in\N,\,i<n} \left(\sum_{j\in\N,\, j<n,\, k_1,\ldots,k_j\in\N,\, k_1,\ldots,k_j\ge 1,\, k_1+\cdots+k_j=i} a_j b_{k_1} \cdots b_{k_j} \right) X^i \\
    % &= \sum_{i\in\N,\,i<n} \left( \sum_{j\in\N,\,j<n} [X^i] a_j b(X)^j \right) X^i \\
    % &= \sum_{j\in\N,\,j<n} \left( \sum_{i\in\N,\,i<n} [X^i] a_j b(X)^j \right) X^i \\
    % &\equiv \sum_{j\in\N,\,j<n} a_j b(X)^j \pmod{X^n}
    (b \circ a)(X)
    &\equiv \sum_{i\in\N,\,i<n} \left(\sum_{k\in\N,\, j_1,\ldots,j_k\in\N,\, j_1,\ldots,j_k\ge 1,\, j_1+\cdots+j_k=i} b_k a_{j_1} \cdots a_{j_k} \right) X^i \\
    &= \sum_{i\in\N,\,i<n} \left(\sum_{k\in\N,\, k<n,\, j_1,\ldots,j_k\in\N,\, j_1,\ldots,j_k\ge 1,\, j_1+\cdots+j_k=i} b_k a_{j_1} \cdots a_{j_k} \right) X^i \\
    &= \sum_{i\in\N,\,i<n} \left( \sum_{k\in\N,\,k<n} [X^i] b_k a(X)^k \right) X^i \\
    &= \sum_{k\in\N,\,k<n} \left( \sum_{i\in\N,\,i<n} [X^i] b_k a(X)^k \right) X^i \\
    &\equiv \sum_{k\in\N,\,k<n} b_k a(X)^k \pmod{X^n}
  \end{align*}
  である.
\end{prf}

合成を「代入」と考えたとき成り立ってほしい性質たちを確認していく.

\begin{prp}
  \label{prp:composition-hom}
  $a(X) \in X R[[X]]$ は環準同型 $a^*\colon R[[X]] \longrightarrow R[[X]]$; $b(X) \longmapsto (b \circ a)(X)$ を定め,
  これは $a^*(X) = a(X)$ を満たす.
  すなわち,$b(X), c(X), d(X) \in R[[X]]$ に対し,
  \begin{enumerate}[(1)]
    \item $b(X) + c(X) = d(X)$ ならば $(b \circ a)(X) + (c \circ a)(X) = (d \circ a)(X)$.
    \item $b(X) c(X) = d(X)$ ならば $(b \circ a)(X) (c \circ a)(X) = (d \circ a)(X)$.
    \item $b(X) = 1$ ならば $(b \circ a)(X) = 1$.
    \item $b(X) = X$ ならば $(b \circ a)(X) = a(X)$.
  \end{enumerate}
\end{prp}

\begin{prf}
  \begin{enumerate}[(1)]
    \item 合成の定義から明らか.
    \item $a(X) = \sum_{i\in\N} a_i X^i$, $b(X) = \sum_{i\in\N} b_i X^i$, $c(X) = \sum_{i\in\N} c_i X^i$ とする.
        $n \in \N$ を任意にとる.
        命題 \ref{prp:composition-mod} より,
        \begin{align*}
          (a \circ d)(X) (b \circ d)(X)
          &\equiv \left( \sum_{i\in\N,\,i<n} a_i d(X)^i \right) \left( \sum_{j\in\N,\,j<n} b_j d(X)^j \right) \\
          &= \sum_{k\in\N,k<2n} \left( \sum_{i,j\in\N,\,i,j<n,\,i+j=k} a_i b_j \right) d(X)^k \\
          &\equiv \sum_{k\in\N,k<n} \left( \sum_{i,j\in\N,\,i+j=k} a_i b_j \right) d(X)^k \\
          &= \sum_{k\in\N,k<n} c_k d(X)^k \\
          &\equiv (c \circ d)(X) \pmod{X^n}
        \end{align*}
        である.
        よって,命題 \ref{prp:mod-limit} より $(a \circ d)(X) (b \circ d)(X) = (c \circ d)(X)$ が従う.
    \item $b_0$ のみ $1$ なので,$(b \circ a)(X) = \sum_{i\in\N} \left( \sum_{0=i} 1 \right) X^i = 1$.
    \item $b_1$ のみ $1$ なので,$(b \circ a)(X) = \sum_{i\in\N} \left( \sum_{k_1\in\N,\,k_1\ge 1,\,k_1=i} a_{k_1} \right) X^i = \sum_{i\in\N} a_i X^i = a(X)$.
  \end{enumerate}
\end{prf}

\begin{prp}
  $a(X), b(X) \in X R[[X]]$ $c(X) \in R[[X]]$ に対し,
  $(c \circ (b \circ a))(X) = ((c \circ b) \circ a)(X)$.
\end{prp}
\begin{prf}
  $c(X) = X$ のとき,命題 \ref{prp:composition-hom} (4) より,
  \[
    (c \circ (b \circ a))(X) = (b \circ a)(X) = ((c \circ b) \circ a)(X)
  \]
  である.
  すなわち $(b \circ a)^*(X) = (a^* \circ b^*)(X)$ である ($(b \circ a)(X)$ は形式的冪級数の合成,$a^* \circ b^*$ は環準同型の合成であることに注意する).

  $R[[X]]$ は $X$ で生成されるので,環準同型は $X$ の行き先で定まる.
  よって $(b \circ a)^* = a^* \circ b^*$ であり,
  任意の $c(X)$ に対し
  \[
    (c \circ (b \circ a))(X) = (b \circ a)^*(c(X)) = (a^* \circ b^*)(c(X)) = ((c \circ b) \circ a)(X)
  \]
  となる.
\end{prf}



これらの理解のもと,$(b \circ a)(X)$ を $b(a(X))$ とも書く.
とくに,$b(0)$ は $b(X)$ の定数項に等しい.

\begin{exm}
  $a(X) = \sum_{i\in\N} a_i X^i \in R[[X]]$ と正の整数 $n$ に対し,
  $a(X^n) = \sum_{i\in\N} a_i X^{ni}$.
\end{exm}

\begin{exm}
  $a(X) = \frac{X}{1 - X} = \sum_{i\in\N} X^{i+1} \in R[[X]]$ と $n \in \N$ に対し,
  $(\underbrace{a \circ \cdots \circ a}_{n})(X) = \frac{X}{1 - n X} = \sum_{i\in\N} n^i X^{i+1}$
  ($n$ 回合成を $a^n$ と書くと $n$ 乗と紛らわしいため避けている).
\end{exm}



\section{合成逆}
\section{微分と積分}
\section{exp}
\section{log}
\section{アルゴリズム}


\end{document}
