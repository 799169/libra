\documentclass{jsarticle}
\usepackage{amsmath,amssymb,amsthm}

\everymath{\displaystyle}
\newcommand{\N}{\mathbb{N}}

\theoremstyle{definition}
\newtheorem{thm}{定理}
\newtheorem*{exm}{例}

\title{メモ (形式的冪級数)}
\author{hos}

\begin{document}
\maketitle

収束のことは考えない.
その代わり,知らない演算をしない.

$\N$ は非負整数全体の集合とする.

\section{形式的冪級数環}
$R$ を可換環とする.

$X$ を不定元として,
$R$ の加算個の直積 $R^\N$ の元 $(a_i)_{i\in\N}$ を形式的に
$\sum_{i\in\N} a_i X^i$ (あるいは $a_0 + a_1 X + a_2 X^2 + \cdots$) と書いたものの集合を $R[[X]]$ とする.
この元を $a(X)$ のように書くこともある.
$a_i$ を $a(X)$ の \textbf{$i$ 次の係数}と呼び,$[X^i] a(X)$ のように書く.
$0$ 次の係数を\textbf{定数項}と呼ぶ.

$\sum_{i\in\N} a_i X^i, \sum_{i\in\N} b_iX^i \in R[[X]]$ に対し,加法と乗法を
\begin{align*}
  \sum_{i\in\N} a_i X^i + \sum_{i\in\N} b_i X^i := \sum_{i\in\N} (a_i + b_i) X^i, \\
  \left(\sum_{i\in\N} a_i X^i\right) \left(\sum_{i\in\N} b_i X^i\right) := \sum_{i\in\N} \left(\sum_{i,j\in\N,\,i+j=k} a_i b_j\right) X^k
\end{align*}
で定め,
$r \in R$ と $\sum_{i\in\N} a_i X^i \in R[[X]]$ に対し,定数倍を
\[
  r \left(\sum_{i\in\N} a_i X^i\right) := \sum_{i\in\N} (r a_i) X^i
\]
で定めると,$R$ 代数 (可換環構造が適切に入る $R$ 加群) になることを示す.
加法の単位元は $0 = 0 + 0 X + 0 X^2 + \cdots$,
乗法の単位元は $1 = 1 + 0 X + 0 X^2 + \cdots$.
乗法の結合法則のみやや非自明で,
$\left(\left(\sum_{i\in\N} a_i X^i\right) \left(\sum_{i\in\N} b_i X^i\right)\right) \left(\sum_{i\in\N} c_i X^i\right)$ の
$m$ 次の係数が
\[
  \sum_{l,k\in\N,\,l+k=m} \left(\sum_{i,j\in\N,\,i+j=k} a_i b_j\right) c_k = \sum_{i,j,k\in\N,\,i+j+k=m} a_i b_j c_k
\]
となることから従う.

この $R$ 代数 $R[[X]]$ を,\textbf{$R$ 係数形式的冪級数環}と呼ぶ.


\section{乗法の逆元}
環の単元とは,乗法の逆元をもつ元のこと.可逆元.$1$ の約数.

\begin{thm}
  $\sum_{i\in\N} a_i X^i \in R[[X]]$ が単元であるための必要十分条件は,$a_0$ が $R$ の単元であること.
\end{thm}
\begin{proof}
  (必要性)
  $\sum_{i\in\N} b_i X^i \in R[[X]]$ が $\left(\sum_{i\in\N} a_i X^i\right) \left(\sum_{i\in\N} b_i X^i\right) = 1$ を満たすとすると,
  定数項を比較して,$a_0 b_0 = 1$ である.

  (十分性)
  $a_0$ が単元のとき,$\sum_{i\in\N} b_i X^i \in R[[X]]$ を
  \begin{align*}
    b_0 &= a_0^{-1}, \\
    b_i &= -a_0^{-1} \sum_{j\in\N,\,1\le j\le i} a_j b_{i-j}
  \end{align*}
  として定めると,$\left(\sum_{i\in\N} a_i X^i\right) \left(\sum_{i\in\N} b_i X^i\right) = 1$ を満たす.
\end{proof}

$a(X) \in R[[X]]$ の乗法の逆元が存在するとき,それは一意なので,
$a(X)^{-1}$ や $\frac{1}{a(X)}$ と書く.

\begin{exm}
  $r \in R$ に対し,$(1 - r X)^{-1} = \sum_{i\in\N} r^i X^i$.
\end{exm}


\end{document}
