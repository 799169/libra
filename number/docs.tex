\documentclass{jsarticle}
\usepackage{amsmath}

\DeclareMathOperator{\Mod}{mod}
\DeclareMathOperator{\xor}{xor}

\title{メモ (整数論)}
\author{hos}

\begin{document}
\maketitle

$\div$ は整数除算 (剰余の符号を被除数に合わせる).

\section{mod 逆元 (2 冪)}
奇数 $a$ に対し,$a \times (3a \xor 2) \equiv 1 \pmod{2^5}$ (黒魔術).

$a b \equiv 1 \pmod{2^k}$ のとき,$a \times b (2 - a b) \equiv 1 \pmod{2^{2k}}$ ($1 - a b (2 - a b) = (1 - a b)^2$ より).

$a b \equiv -1 \pmod{2^k}$ のとき,$a \times b (2 + a b) \equiv -1 \pmod{2^{2k}}$ ($1 + a b (2 + a b) = (1 + a b)^2$ より).

\section{mod 逆元 (一般)}
$(r_0, s_0, t_0) = (a, 1, 0)$, 
$(r_1, s_1, t_1) = (b, 0, 1)$, 
$(r_i, s_i, t_i) = (r_{i-2}, s_{i-2}, t_{i-2}) - (r_{i-2} \div r_{i-1}) (r_{i-1}, s_{i-1}, t_{i-1})$ とすると,
$r_i = a s_i + b t_i$, $\gcd(s_i, t_i) = 1$ が不変.
$r_k = 0$ になったとき,$\lvert r_{k-1} \rvert = \gcd(a, b)$ なので,
特に $a s_{i-1} \equiv \pm 1 \pmod{b}$.

$k \ge 3$ なら,
$|s_2| < |s_3| < \dots < |s_{k-1}| < |s_k| = \dfrac{\lvert b \rvert}{\gcd(a, b)}$, 
$|t_2| < |t_3| < \dots < |t_{k-1}| < |t_k| = \dfrac{\lvert a \rvert}{\gcd(a, b)}$.

\section{Montgomery reduction}
正の奇数 $M$ に対し,$M < 2^k$ として,$M' \equiv -M^{-1} \pmod{2^k}$ をとっておく.

整数 $a$ に対し,
$2^{-k} a \equiv \dfrac{a + (a M' \bmod 2^k) M}{2^k}$ (分子が $\equiv 0 \pmod{2^k}$ かつ $\equiv a \pmod{M}$ なので),
$0 \le a < M$ なら右辺は $0$ 以上 $2 M$ 未満.

$\Mod M$ で加減乗をたくさん行うとき,$f(a) = 2^k a \bmod M$ で変換してから行う.$f(a b) \equiv 2^{-k} f(a) f(b) \pmod{M}$.
$2$ 冪以外での除算は $f$ の適用時のみになる.

\section{多項式除算}
除算 $e(t) = f(t) q(t) + r(t)$ ($\deg e = m$, $\deg f = n$, $m \ge n$, $\deg q = m - n$, $\deg r < n$) の
両辺を $t^m$ で割って $T = t^{-1}$ とすると,
$E(T) = F(T) Q(T) + T^{m-n+1} R(T)$.
ここで $E$, $F$, $Q$, $R$ はそれぞれ $e$, $f$, $q$, $r$ を係数逆順にした多項式 ($r$ は $n$ 項まで $0$ 埋め).

$f$ が monic なら $F(0) \ne 0$ なので $F(T) F'(T) \equiv 1 \pmod{T^{m-n+1}}$ なる $F'$ がとれて,
$Q(T) = E(T) F'(T) \bmod T^{m-n+1}$ として $Q$ が求まる.

$\Mod f(t)$ を常にとりながら加減乗を行うとき,被除数は次数 $2 n - 2$ 以下なので,
$F'$ は $\Mod T^{n-1}$ で $1$ 回求めておけばよい.
$F'$ は $F'(T) = 1$ から $F' \mapsto F' (2 - F F')$ を $\lceil \log_2 (n - 1) \rceil$ 回繰り返せば求まる.
$O(n (\log n)^2)$ 時間だが,FFT の配列の長さをちゃんとやると $O(n \log n)$ 時間にできる.



\end{document}
