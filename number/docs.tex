\documentclass{jsarticle}
\usepackage{amsmath,amssymb}

\DeclareMathOperator{\Mod}{mod}
\DeclareMathOperator{\xor}{xor}
\newcommand{\F}{\mathbb{F}}

\title{メモ (整数論)}
\author{hos}

\begin{document}
\maketitle

$\div$ は整数除算 (剰余の符号を被除数に合わせる).


\section{mod 逆元 (2 冪)}
奇数 $a$ に対し,$a \times (3a \xor 2) \equiv 1 \pmod{2^5}$ (黒魔術).

$a b \equiv 1 \pmod{2^k}$ のとき,$a \times b (2 - a b) \equiv 1 \pmod{2^{2k}}$ ($1 - a b (2 - a b) = (1 - a b)^2$ より).

$a b \equiv -1 \pmod{2^k}$ のとき,$a \times b (2 + a b) \equiv -1 \pmod{2^{2k}}$ ($1 + a b (2 + a b) = (1 + a b)^2$ より).


\section{mod 逆元 (一般)}
$(r_0, s_0, t_0) = (a, 1, 0)$, 
$(r_1, s_1, t_1) = (b, 0, 1)$, 
$(r_i, s_i, t_i) = (r_{i-2}, s_{i-2}, t_{i-2}) - (r_{i-2} \div r_{i-1}) (r_{i-1}, s_{i-1}, t_{i-1})$ とすると,
$r_i = a s_i + b t_i$, $\gcd(s_i, t_i) = 1$ が不変.
$r_k = 0$ になったとき,$\lvert r_{k-1} \rvert = \gcd(a, b)$ なので,
特に $a s_{i-1} \equiv \pm 1 \pmod{b}$.

$k \ge 3$ なら,
$|s_2| < |s_3| < \dots < |s_{k-1}| < |s_k| = \dfrac{\lvert b \rvert}{\gcd(a, b)}$, 
$|t_2| < |t_3| < \dots < |t_{k-1}| < |t_k| = \dfrac{\lvert a \rvert}{\gcd(a, b)}$.


\section{mod 平方根 (素数)}
$p$ を奇素数とする.
平方剰余 $a \in \F_p^\times$ に対し,
$b^2 - a$ が平方非剰余となる $b \in \F_p$ は $\dfrac{p - 1}{2}$ 個ある ($b^2 - a = c^2$ の解は
$(b + c) (b - c) = a$ より $(b, c) = \left(\dfrac{t + a t^{-1}}{2}, \dfrac{t - a t^{-1}}{2}\right)$ と書ける $p - 1$ 個で,
$c$ を $-c$ にしても同じ $b$ が対応して,$c = 0$ は解でないので).
よってそのような $b$ は期待値約 $2$ 回の乱択で見つかる.

$2$ 次体 $\F_p(\sqrt{b^2 - a})$ を考えて,$x = \left(b + \sqrt{b^2 - a}\right)^{\frac{p+1}{2}}$ とすると,
Frobenius 準同型の性質より $x^2 = \left(b + \sqrt{b^2 - a}\right) \left(b + \sqrt{b^2 - a}\right)^p = \left(b + \sqrt{b^2 - a}\right) \left(b - \sqrt{b^2 - a}\right) = a$.
$x^2 = a$ の解は $\F_p(\sqrt{b^2 - a})$ においても $2$ 個しかないので,$x \in \F_p$ である.


\section{連立合同式}
$t \equiv B \pmod{M}$ かつ $a t \equiv b \pmod{m}$ なる $t$ を求める.
$t = B + M z$ として,$a M z \equiv b - a B \pmod{m}$ が条件.
$g = \gcd(a M, m)$ とおいて,$g \nmid b - a B$ なら解なし.
そうでないとき,
$x \equiv \left( \dfrac{a M}{g} \right)^{-1} \pmod{\dfrac{m}{g}}$ として (互除法で $a M x + m y = g$ なる $(x, y)$ も求まっている),
$z \equiv x \dfrac{b - a B}{g} \pmod{\dfrac{m}{g}}$.
$t$ は $\Mod \dfrac{M m}{g}$ で一意.


\section{Montgomery reduction}
正の奇数 $M$ に対し,$M < 2^k$ として,$M' \equiv -M^{-1} \pmod{2^k}$ をとっておく.

整数 $a$ に対し,
$2^{-k} a \equiv \dfrac{a + (a M' \bmod 2^k) M}{2^k}$ である (分子が $\equiv 0 \pmod{2^k}$ かつ $\equiv a \pmod{M}$ なので).
$0 \le a < 2^k M$ なら右辺は $0$ 以上 $2 M$ 未満.

$\Mod M$ で加減乗をたくさん行うとき,$f(a) = 2^k a \bmod M$ で変換してから行う.$f(a b) \equiv 2^{-k} f(a) f(b) \pmod{M}$.
$2$ 冪以外での除算は $f$ の適用時のみになる.


\section{多項式除算}
除算 $e(t) = f(t) q(t) + r(t)$ ($\deg e = m$, $\deg f = n$, $m \ge n$, $\deg q = m - n$, $\deg r < n$) の
両辺を $t^m$ で割って $T = t^{-1}$ とすると,
$E(T) = F(T) Q(T) + T^{m-n+1} R(T)$.
ここで $E$, $F$, $Q$, $R$ はそれぞれ $e$, $f$, $q$, $r$ を係数逆順にした多項式 ($r$ は $n$ 項まで $0$ 埋め).

$f$ が monic なら $F(0) \ne 0$ なので $F(T) F'(T) \equiv 1 \pmod{T^{m-n+1}}$ なる $F'$ がとれて,
$Q(T) = E(T) F'(T) \bmod T^{m-n+1}$ として $Q$ が求まる.

$\Mod f(t)$ を常にとりながら加減乗を行うとき,被除数は次数 $2 n - 2$ 以下なので,
$F'$ は $\Mod T^{n-1}$ で $1$ 回求めておけばよい.
$F'$ は $F'(T) = 1$ から $F' \mapsto F' (2 - F F')$ を $\lceil \log_2 (n - 1) \rceil$ 回繰り返せば求まる.
$O(n (\log n)^2)$ 時間だが,FFT の配列の長さをちゃんとやると $O(n \log n)$ 時間にできる.


\end{document}
